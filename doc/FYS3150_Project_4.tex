\documentclass[reprint,english,notitlepage]{revtex4-1}  % defines the basic parameters of the document

% if you want a single-column, remove reprint

% allows special characters (including æøå)
\usepackage[utf8]{inputenc}
\usepackage[english]{babel}

%% note that you may need to download some of these packages manually, it depends on your setup.
%% I recommend downloading TeXMaker, because it includes a large library of the most common packages.

\usepackage{physics,amssymb}  % mathematical symbols (physics imports amsmath)
\usepackage{graphicx}         % include graphics such as plots
\usepackage{xcolor}           % set colors
\usepackage{hyperref}         % automagic cross-referencing (this is GODLIKE)
\usepackage{tikz}             % draw figures manually
\usepackage{listings}         % display code
\usepackage{subfigure}        % imports a lot of cool and useful figure commands
\usepackage{cprotect}
\usepackage{float}


% defines the color of hyperref objects
% Blending two colors:  blue!80!black  =  80% blue and 20% black
\hypersetup{ % this is just my personal choice, feel free to change things
    colorlinks,
    linkcolor={red!50!black},
    citecolor={blue!50!black},
    urlcolor={blue!80!black}}

%% Defines the style of the programming listing
%% This is actually my personal template, go ahead and change stuff if you want
\lstnewenvironment{python}{
	\lstset{ %
		inputpath=,
		backgroundcolor=\color{white!88!black},
		basicstyle={\ttfamily\scriptsize},
		commentstyle=\color{magenta},
		language=Python,
		morekeywords={True,False},
		tabsize=4,
		stringstyle=\color{green!55!black},
		frame=single,
		keywordstyle=\color{blue},
		showstringspaces=false,
		columns=fullflexible,
		keepspaces=true}
}{}

\lstnewenvironment{cpp}{
	\lstset{ %
		inputpath=,
		backgroundcolor=\color{white!88!black},
		basicstyle={\ttfamily\scriptsize},
		commentstyle=\color{magenta},
		language=C++,
		morekeywords={True,False},
		tabsize=4,
		stringstyle=\color{green!55!black},
		frame=single,
		keywordstyle=\color{blue},
		showstringspaces=false,
		columns=fullflexible,
		keepspaces=true}
}{}

\lstset{literate=
  {á}{{\'a}}1 {é}{{\'e}}1 {í}{{\'i}}1 {ó}{{\'o}}1 {ú}{{\'u}}1
  {Á}{{\'A}}1 {É}{{\'E}}1 {Í}{{\'I}}1 {Ó}{{\'O}}1 {Ú}{{\'U}}1
  {à}{{\`a}}1 {è}{{\`e}}1 {ì}{{\`i}}1 {ò}{{\`o}}1 {ù}{{\`u}}1
  {À}{{\`A}}1 {È}{{\'E}}1 {Ì}{{\`I}}1 {Ò}{{\`O}}1 {Ù}{{\`U}}1
  {ä}{{\"a}}1 {ë}{{\"e}}1 {ï}{{\"i}}1 {ö}{{\"o}}1 {ü}{{\"u}}1
  {Ä}{{\"A}}1 {Ë}{{\"E}}1 {Ï}{{\"I}}1 {Ö}{{\"O}}1 {Ü}{{\"U}}1
  {â}{{\^a}}1 {ê}{{\^e}}1 {î}{{\^i}}1 {ô}{{\^o}}1 {û}{{\^u}}1
  {Â}{{\^A}}1 {Ê}{{\^E}}1 {Î}{{\^I}}1 {Ô}{{\^O}}1 {Û}{{\^U}}1
  {œ}{{\oe}}1 {Œ}{{\OE}}1 {æ}{{\ae}}1 {Æ}{{\AE}}1 {ß}{{\ss}}1
  {ű}{{\H{u}}}1 {Ű}{{\H{U}}}1 {ő}{{\H{o}}}1 {Ő}{{\H{O}}}1
  {ç}{{\c c}}1 {Ç}{{\c C}}1 {ø}{{\o}}1 {å}{{\r a}}1 {Å}{{\r A}}1
  {€}{{\euro}}1 {£}{{\pounds}}1 {«}{{\guillemotleft}}1
  {»}{{\guillemotright}}1 {ñ}{{\~n}}1 {Ñ}{{\~N}}1 {¿}{{?`}}1
}



\usepackage{thmtools}
\DeclareMathOperator{\nullspace}{Nul}
\DeclareMathOperator{\collspace}{Col}
\DeclareMathOperator{\rref}{Rref}
%%\DeclareMathOperator{\dim}{Dim}

 % "meq": must be equal
\newcommand{\meq}{\overset{!}{=}}
\newcommand\numberthis{\addtocounter{equation}{1}\tag{\theequation}}

\newcommand{\R}{\mathbb{R}}
\newcommand*\Heq{\ensuremath{\overset{\kern2pt L'H}{=}}}
\usepackage{bm}
\newcommand{\uveci}{{\bm{\hat{\textnormal{\bfseries\i}}}}}
\newcommand{\uvecj}{{\bm{\hat{\textnormal{\bfseries\j}}}}}
\DeclareRobustCommand{\uvec}[1]{{%
  \ifcsname uvec#1\endcsname
     \csname uvec#1\endcsname
   \else
    \bm{\hat{\mathbf{#1}}}%
   \fi
}}
\usepackage[binary-units=true]{siunitx}

\makeatletter
\newcommand*{\balancecolsandclearpage}{%
  \close@column@grid
  \cleardoublepage
  \twocolumngrid
}
\makeatother

\newcounter{subproject}
\renewcommand{\thesubproject}{\alph{subproject}}
\newenvironment{subproj}{
\begin{description}
	\item[\refstepcounter{subproject}(\thesubproject)]
}{\end{description}}


\begin{document}
\title{Title}   % self-explanatory
\author{Eivind Støland, Anders P. Åsbø}               % self-explanatory
\date{\today}                             % self-explanatory
\noaffiliation                            % ignore this

\begin{abstract}
Abstract
\end{abstract}

\maketitle                                % creates the title, author, date


\tableofcontents

\section{Introduction} \label{sec:I}


\newpage

\section{Formalism} \label{sec:II}

\subsection{The Ising Model} \label{sec:II:a}

The Ising model is a mathematical model of ferromagnetic systems. It is based on a set of particles in a lattice with either spin up or down, and the energy that a particle has is dependent on other adjacent particles and on an external magnetic field if there is one. The energy of a configuration of such a system is thus:

\begin{align*}
E &= -J \sum\limits_{<kl>}^N s_k s_l - B \sum\limits_k^N s_k \numberthis \label{eq:ising_general_total_energy} \, ,
\end{align*}

where $J$ is a parameter determining the strength of the interaction between adjacent particles, $B$ is a parameter determining the strength of the external magnetic field, and $N$ is the number of particles. The notation $<kl>$ in the first sum indicates that we should sum over all adjacent particles, and the values summed over, $s$, are the spins of particles, where the subscripts denote which particle it belongs to. The spins can be $s = \pm 1$, which we often denote with an arrow pointing upwards ($\uparrow$) for $s = +1$ and a downwards pointing arrow ($\downarrow$) for $s = -1$.

In order to further proceed we need a probability that governs the system. For this purpose, the Boltzmann distribution is used:

\begin{align*}
P_i (\beta) &= \frac{e^{-\beta E_i}}{Z} \numberthis \label{eq:boltzmann_dist} \, ,
\end{align*}

where $\beta = (k_B T)^{-1}$, $E_i$ the energy of a specific configuration of the system, $Z$ is the partition function, and the subscript $i$ denotes the configuration of the system. The parameter $\beta$ contains $k_B$ which is the Boltzmann constant, and $T$ is the temperature. The partition function is a normalization factor for the probability distribution, and is the sum of all the possible Boltzmann factors:

\begin{align*}
Z &= \sum\limits_i^M e^{-\beta E_i} \numberthis \label{eq:ising_partition_function} \, ,
\end{align*}  

where $M$ is the amount of microstates of the system. The magnetization of a given configuration is given by the sum of all the spins:

\begin{align*}
\mathcal{M}_i &= \sum\limits_j^N s_j \numberthis \label{eq:ising_magnetization}
\end{align*}

The mean of a general variable, $Q_i$, can be defined as:

\begin{align*}
\langle Q \rangle &= \sum\limits_i^M Q_i P_i(\beta) 
\end{align*}
We will also need the relation between the Helmholtz free energy $F$ and the partition function:

\begin{align*}
F &= -k_B T \ln (Z) \, ,\numberthis \label{eq:helmholtz_free_energy}
\end{align*}

and also how heat capacity $C_V$ at constant volume relates to the partition function:

\begin{align*}
C_V &= \frac{1}{k_B T^2} (\langle E^2 \rangle - \langle E \rangle^2) \numberthis \label{eq:heat_capacity}
\end{align*}

We will also need the magnetic susceptibility, $\chi$, of the system:

\begin{align*}
\chi &= \frac{1}{k_B T} ( \langle \mathcal{M}^2 \rangle - \langle \mathcal{M} \rangle^2) \numberthis \label{eq:magnetic_susceptibility}
\end{align*}

In the following section we look at a sample system.


\subsubsection{Periodic 2x2 square lattice with no external field}

We look at a system where we have particles in a 2x2 square lattice. A configuration of the system can be visualized as follows: \newline

\begin{center}
\begin{tabular}{cc}
$\bullet$ & $\bullet$ \\
$\bullet$ & $\bullet$
\end{tabular}  , \newline
\end{center}

where we change the dots out for arrows to mark the spin of the particles. A sample configuration of the system can be visualized as follows: \newline

\begin{center}
\begin{tabular}{cc}
$\uparrow$ & $\downarrow$ \\
$\downarrow$ & $\uparrow$
\end{tabular}  . \newline
\end{center}


We can denote the spins of the system as $s_{ij}$, where the index $i$ corresponds to the row the particle is in and $j$ to the column. With periodic boundary conditions, the total energy of the system is then given by:

\begin{align*}
E &= -J ( 2s_{11}s_{12} + 2s_{11}s_{21} + 2s_{22}s_{21} + 2s_{22}s_{12}) \numberthis \label{eq:2x2_sq_lattice_en}
\end{align*}

As there are two possible states that the particles can be in, and there are four particles total, there are $2^4 = 16$ microstates. We are interested in the possible energies of this system, and they are listed in table \ref{table:2x2_sq_lattice_en_and_mag} along with the total magnetization, where we have used \eqref{eq:2x2_sq_lattice_en} and \eqref{eq:ising_magnetization} to calculate these values. Note that the values in this table are sorted by macrostate, as some of the microstates are effectively equal because of the symmetries of the system. There are a total of 6 configurations with two spins pointing upwards, but these do not all have the same energy, which means that the macrostate of the system cannot be uniquely determined by the amount of spins that point up (or down for that matter). If the two spins pointing upwards are adjacent the total energy is 0, and there are four such configurations. If the two spins are not adjacent the total energy is $8J$, and there are two such states. As there are then two macrostates (different energies) corresponding to two spins pointing up, we need a second identifier. The degeneracy of these macrostates differ, and so by listing both the degeneracy and the amount of spins pointing up we can uniquely determine all the macrostates of the system. This was used in the table.

\begin{table}[H]
\centering
\caption{This table contains the energies and total magnetization for all possible macrostates of the Ising model with a 2x2 square lattice of particles. The uniqueness of the macro state is determined by the amount of spins that are up (first column) and the degeneracy of said state (second column). Both are needed, as in the case with two spins pointing up there are four configurations with the two upwards pointing spins being adjacent, and two with them not being adjacent.} \label{table:2x2_sq_lattice_en_and_mag}
\begin{tabular}{|c|c|c|c|}
\hline
Spins up & Degeneracy & Energy & Magnetization \\
\hline
4 & 1 & -8J & 4 \\
3 & 4 & 0 & 2 \\
2 & 4 & 0 & 0 \\
2 & 2 & 8J & 0 \\
1 & 4 & 0 & -2 \\
0 & 1 & -8J & -4 \\
\hline
\end{tabular}
\end{table}


We can calculate the partition function of the system using \eqref{eq:ising_partition_function}:

\begin{align*}
Z &= \sum\limits_i^M e^{-\beta E_i} \\
&= 2 \bigg(e^{8J\beta} + e^{-8J\beta} \bigg) + 12 \\
&= 4\cosh(8J\beta) + 12
\end{align*}

The probability distribution of the system is thus:

\begin{align*}
P_i(\beta) &= \frac{e^{-\beta E_i}}{4\cosh (8J\beta) + 12}
\end{align*}

The expectation value for the energy can be found as follows:

\begin{align*}
\langle E \rangle &= \sum\limits_i^M E_i P_i(\beta) \\
 &= -8J \frac{2e^{8J\beta}}{4\cosh (8J\beta) + 12} + 8J \frac{2e^{-8J}}{4\cosh (8J\beta) + 12} \\
 &= -8J\frac{2(e^{8J\beta} - e^{-8J\beta})}{4\cosh (8J\beta) + 12} \\
 &= -8J\frac{\sinh(8J\beta) }{\cosh (8J\beta) + 3} \numberthis \label{eq:2x2_energy}
\end{align*}

It is simple to see that $\langle M \rangle = 0$, so a more interesting measure would be the expectation value of the absolute of the magnetization. The expectation value for the absolute value of the magnetization can be found similarly:

\begin{align*}
\langle |\mathcal{M}| \rangle &= \sum\limits_i^M |\mathcal{M}_i| P_i(\beta) \\
 &= \frac{2e^{8J\beta} + 4}{\cosh (8 J \beta) + 3} \numberthis \label{eq:2x2_abs_magnetization}
\end{align*}

In order to find the heat capacity it is first necessary to find the expectation value of squared energy:

\begin{align*}
\langle E^2 \rangle &= \sum\limits_i^M E_i^2 P_i(\beta) \\
&= 64J^2 \frac{\cosh(8J\beta)}{\cosh(8J\beta) + 3}
\end{align*}

This gives us the heat capacity by equation \eqref{eq:heat_capacity}:

\begin{align*}
C_V &= \frac{1}{k_B T^2} ( \langle E^2 \rangle - \langle E \rangle^2) \\
&= \frac{64J^2}{k_B T^2} \frac{e^{-8J\beta}}{\cosh(8J\beta) + 3} \numberthis \label{eq:2x2_heat_capacity}
\end{align*}

Lastly we also want to find the magnetic susceptibility of the system. In order to do that we need to find:

\begin{align*}
\langle \mathcal{M}^2 \rangle &= \sum\limits_i \mathcal{M}_i^2 P_i(\beta) \\
&= 8 \frac{e^{8J\beta} + 1}{\cosh(8J\beta) + 3}
\end{align*}

This gives us the susceptibility by using equation \eqref{eq:magnetic_susceptibility}:

\begin{align*}
\chi &= \frac{1}{k_B T} ( \langle \mathcal{M}^2 \rangle - \langle \mathcal{M} \rangle^2 ) \\
&= \frac{8}{k_B T} \frac{e^{8J \beta}}{\cosh( 8J \beta) + 3} \numberthis \label{eq:2x2_susceptibility}
\end{align*}

All of these values we have calculated here can be compared with numerical results later on in order to evaluate the veracity of the numerical results.

The difference in energy between two states of a two dimensional lattice, can be calculated using \eqref{eq:ising_general_total_energy}. Assuming no outside magnetic field gives us the difference between states \(1\) and \(2\) as
\begin{align*}
	\Delta E = E_{2} - E_{1} = \left(-J \sum\limits_{<kl>}^N s_k^{2} s_l^{2}\right) -
	\left(-J \sum\limits_{<kl>}^N s_k^{1} s_l^{1}\right),
\end{align*}
where a superscript is used to denote which state the respective spins belong to.

Since we are only interested in the difference in energy from flipping a single spin in the lattice, we can assume that \(s_{k}^{1} = s_{k}^{2} = s_{k}\). Thus we can rewrite \(\Delta E\) as
\begin{align*}
 	\Delta E = J \sum_{<kl>}^{N} s_{k} \left(s_{l}^{1} - s_{l}^{2}\right).
\end{align*}

Furthermore, any spin can only be \(\pm 1\) and spin \(s_{l}\) is flipped such that \(s_{l}^{2} = -s_{l}^{1}\). From this it follows that \begin{align*}
	\left(s_{l}^{1} - s_{l}^{2}\right) = \left(s_{l}^{1} + s_{l}^{1}\right) = 2s_{l}^{1}.
\end{align*}

The difference in energy can then be written as

\begin{align*}
 	\Delta E = 2Js_{l}^{1} \sum_{<k>}^{N} s_{k}, \numberthis \label{eq:delta_e}
\end{align*}

where we only sum over \(k\), since only one spin \(s_{l}\) is flipped.
For a two dimensional lattice, each spin only has four neighbours, thus we can write out the sum for any randomly chosen \(s_{l}\) with neighbours \(s_{a}, s_{b}, s_{c}, s_{d}\) as

\begin{align*}
	\Delta E = 2Js_{l}^{1}\left(s_{a} + s_{b} + s_{c} + s_{d}\right) \\
	\Delta E = 2J(\pm 1)\left[(\pm 1) + (\pm 1) + (\pm 1) + (\pm 1)\right]
\end{align*}

From this we see that if all spins \(s_{k}\) have the same sign we get
\begin{align*}
	\Delta E = \pm 2J(\pm 4) = \pm 8J.
\end{align*}
If three of the spins \(s_{k}\) have the same sign we get
\begin{align*}
	\Delta E = \pm 2J(\pm 2) = \pm 4J.
\end{align*}
If two of the spins \(s_{k}\) have the same sign we get
\begin{align*}
	\Delta E = \pm 2J(0) = 0.
\end{align*}
Thus we only have five possible diffrences in energy between states when we only flip one spin.

The difference in magnetization can be simillarly found using \eqref{eq:ising_magnetization}, and subtracting \(M_{2} - M_{1}\)
\begin{align*}
	\mathcal{M}_2 - \mathcal{M}_{1} &= \sum\limits_j^N s_j^{2} - \sum\limits_j^N s_j^{1} =
	\sum\limits_{j}^{N}\left(s_{j}^{2} - s_{j}^{1}\right).
\end{align*}
Because we assume only spin \(s_{l}\) is flipped when changing states, the difference in magnetization becomes
\begin{align*}
 	\mathcal{M}_2 - \mathcal{M}_{1} = \left(s_{l}^{2} - s_{l}^{1}\right).
\end{align*} 
Using that \(s_{l}^{1} = - s_{l}^{2}\), we get
\begin{align*}
	\mathcal{M}_2 - \mathcal{M}_{1} = \left(s_{l}^{2} + s_{l}^{2}\right) = 2s_{l}^{2},
\end{align*}
which gives that
\begin{align*}
	\mathcal{M}_2 = \mathcal{M}_1 + 2s_{l}^{2} \numberthis \label{eq:delta_m}
\end{align*}

\newpage

\section{Method} \label{sec:III}


\newpage

\section{Results} \label{sec:IV}


\newpage

\section{Discussion} \label{sec:V}


\newpage

\section{Conclusion} \label{sec:VI}


\onecolumngrid
\bibliography{kilder.bib}{}
\newpage
\twocolumngrid

\appendix
\section{Source code} \label{A}
All code for this report was written in C++ and Python 3.8, and the complete set of files can be found at
\url{https://github.com/eivinsto/FYS3150_Project_X.git}.

\newpage
\section{Selected results} \label{B}
Here is a folder of selected results from running our code.

\url{https://github.com/eivinsto/FYS3150_Project_X/tree/master/data}

~
\newpage
\section{System specifications} \label{C}
All results included in this report were achieved by running the implementation on the following system:

\begin{itemize}
	\item CPU: AMD Ryzen \(9\) \(3900\)X
	\item RAM: \(2\times\SI{8}{\giga\byte}\) Corsair Vengeance LPX DDR\(4\) \(\SI{3200}{\mega\hertz}\)
\end{itemize}

\end{document}
