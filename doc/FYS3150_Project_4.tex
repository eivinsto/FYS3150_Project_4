\documentclass[reprint,english,notitlepage]{revtex4-1}  % defines the basic parameters of the document

% if you want a single-column, remove reprint

% allows special characters (including æøå)
\usepackage[utf8]{inputenc}
\usepackage[english]{babel}

%% note that you may need to download some of these packages manually, it depends on your setup.
%% I recommend downloading TeXMaker, because it includes a large library of the most common packages.

\usepackage{physics,amssymb}  % mathematical symbols (physics imports amsmath)
\usepackage{graphicx}         % include graphics such as plots
\usepackage{xcolor}           % set colors
\usepackage{hyperref}         % automagic cross-referencing (this is GODLIKE)
\usepackage{tikz}             % draw figures manually
\usepackage{listings}         % display code
\usepackage{subfigure}        % imports a lot of cool and useful figure commands
\usepackage{cprotect}
\usepackage{float}


% defines the color of hyperref objects
% Blending two colors:  blue!80!black  =  80% blue and 20% black
\hypersetup{ % this is just my personal choice, feel free to change things
    colorlinks,
    linkcolor={red!50!black},
    citecolor={blue!50!black},
    urlcolor={blue!80!black}}

%% Defines the style of the programming listing
%% This is actually my personal template, go ahead and change stuff if you want
\lstnewenvironment{python}{
	\lstset{ %
		inputpath=,
		backgroundcolor=\color{white!88!black},
		basicstyle={\ttfamily\scriptsize},
		commentstyle=\color{magenta},
		language=Python,
		morekeywords={True,False},
		tabsize=4,
		stringstyle=\color{green!55!black},
		frame=single,
		keywordstyle=\color{blue},
		showstringspaces=false,
		columns=fullflexible,
		keepspaces=true}
}{}

\lstnewenvironment{cpp}{
	\lstset{ %
		inputpath=,
		backgroundcolor=\color{white!88!black},
		basicstyle={\ttfamily\scriptsize},
		commentstyle=\color{magenta},
		language=C++,
		morekeywords={True,False},
		tabsize=4,
		stringstyle=\color{green!55!black},
		frame=single,
		keywordstyle=\color{blue},
		showstringspaces=false,
		columns=fullflexible,
		keepspaces=true}
}{}

\lstset{literate=
  {á}{{\'a}}1 {é}{{\'e}}1 {í}{{\'i}}1 {ó}{{\'o}}1 {ú}{{\'u}}1
  {Á}{{\'A}}1 {É}{{\'E}}1 {Í}{{\'I}}1 {Ó}{{\'O}}1 {Ú}{{\'U}}1
  {à}{{\`a}}1 {è}{{\`e}}1 {ì}{{\`i}}1 {ò}{{\`o}}1 {ù}{{\`u}}1
  {À}{{\`A}}1 {È}{{\'E}}1 {Ì}{{\`I}}1 {Ò}{{\`O}}1 {Ù}{{\`U}}1
  {ä}{{\"a}}1 {ë}{{\"e}}1 {ï}{{\"i}}1 {ö}{{\"o}}1 {ü}{{\"u}}1
  {Ä}{{\"A}}1 {Ë}{{\"E}}1 {Ï}{{\"I}}1 {Ö}{{\"O}}1 {Ü}{{\"U}}1
  {â}{{\^a}}1 {ê}{{\^e}}1 {î}{{\^i}}1 {ô}{{\^o}}1 {û}{{\^u}}1
  {Â}{{\^A}}1 {Ê}{{\^E}}1 {Î}{{\^I}}1 {Ô}{{\^O}}1 {Û}{{\^U}}1
  {œ}{{\oe}}1 {Œ}{{\OE}}1 {æ}{{\ae}}1 {Æ}{{\AE}}1 {ß}{{\ss}}1
  {ű}{{\H{u}}}1 {Ű}{{\H{U}}}1 {ő}{{\H{o}}}1 {Ő}{{\H{O}}}1
  {ç}{{\c c}}1 {Ç}{{\c C}}1 {ø}{{\o}}1 {å}{{\r a}}1 {Å}{{\r A}}1
  {€}{{\euro}}1 {£}{{\pounds}}1 {«}{{\guillemotleft}}1
  {»}{{\guillemotright}}1 {ñ}{{\~n}}1 {Ñ}{{\~N}}1 {¿}{{?`}}1
}



\usepackage{thmtools}
\DeclareMathOperator{\nullspace}{Nul}
\DeclareMathOperator{\collspace}{Col}
\DeclareMathOperator{\rref}{Rref}
%%\DeclareMathOperator{\dim}{Dim}

 % "meq": must be equal
\newcommand{\meq}{\overset{!}{=}}
\newcommand\numberthis{\addtocounter{equation}{1}\tag{\theequation}}

\newcommand{\R}{\mathbb{R}}
\newcommand*\Heq{\ensuremath{\overset{\kern2pt L'H}{=}}}
\usepackage{bm}
\newcommand{\uveci}{{\bm{\hat{\textnormal{\bfseries\i}}}}}
\newcommand{\uvecj}{{\bm{\hat{\textnormal{\bfseries\j}}}}}
\DeclareRobustCommand{\uvec}[1]{{%
  \ifcsname uvec#1\endcsname
     \csname uvec#1\endcsname
   \else
    \bm{\hat{\mathbf{#1}}}%
   \fi
}}
\usepackage[binary-units=true]{siunitx}

\makeatletter
\newcommand*{\balancecolsandclearpage}{%
  \close@column@grid
  \cleardoublepage
  \twocolumngrid
}
\makeatother

\newcounter{subproject}
\renewcommand{\thesubproject}{\alph{subproject}}
\newenvironment{subproj}{
\begin{description}
	\item[\refstepcounter{subproject}(\thesubproject)]
}{\end{description}}


\begin{document}
\title{Title}   % self-explanatory
\author{Eivind Støland, Anders P. Åsbø}               % self-explanatory
\date{\today}                             % self-explanatory
\noaffiliation                            % ignore this

\begin{abstract}
Abstract
\end{abstract}

\maketitle                                % creates the title, author, date


\tableofcontents

\section{Introduction} \label{sec:I}


\newpage

\section{Formalism} \label{sec:II}

\subsection{The Ising Model} \label{sec:II:a}

The Ising model is a mathematical model of ferromagnetic systems. It is based on a set of particles in a lattice with either spin up or down, and the energy that a particle has is dependent on other adjacent particles and on an external magnetic field if there is one. We will be working with this model in the canonical ensemble. The energy of a configuration of such a system is:

\begin{align*}
E &= -J \sum\limits_{<kl>}^N s_k s_l - B \sum\limits_k^N s_k \numberthis \label{eq:ising_general_total_energy} \, ,
\end{align*}

where $J$ is a parameter determining the strength of the interaction between adjacent particles, $B$ is a parameter determining the strength of the external magnetic field, and $N$ is the number of particles. The notation $<kl>$ in the first sum indicates that we should sum over all adjacent particles, and the values summed over, $s$, are the spins of particles, where the subscripts denote which particle it belongs to. The spins can be $s = \pm 1$, which we often denote with an arrow pointing upwards ($\uparrow$) for $s = +1$ and a downwards pointing arrow ($\downarrow$) for $s = -1$.

In order to further proceed we need a probability that governs the system. For this purpose, the Boltzmann distribution is used:

\begin{align*}
P_i (\beta) &= \frac{e^{-\beta E_i}}{Z} \numberthis \label{eq:boltzmann_dist} \, ,
\end{align*}

where $\beta = (k_B T)^{-1}$, $E_i$ the energy of a specific configuration of the system, $Z$ is the partition function, and the subscript $i$ denotes the configuration of the system. The parameter $\beta$ contains $k_B$ which is the Boltzmann constant, and $T$ is the temperature. The partition function is a normalization factor for the probability distribution, and is the sum of all the possible Boltzmann factors:

\begin{align*}
Z &= \sum\limits_i^M e^{-\beta E_i} \numberthis \label{eq:ising_partition_function} \, ,
\end{align*}  

where $M$ is the amount of microstates of the system. The magnetization of a given configuration is given by the sum of all the spins:

\begin{align*}
\mathcal{M}_i &= \sum\limits_j^N s_j \numberthis \label{eq:ising_magnetization}
\end{align*}

The mean of a general variable, $Q_i$, can be defined as:

\begin{align*}
\langle Q \rangle &= \sum\limits_i^M Q_i P_i(\beta) 
\end{align*}
We will also need the relation between the Helmholtz free energy $F$ and the partition function:

\begin{align*}
F &= -k_B T \ln (Z) \, ,\numberthis \label{eq:helmholtz_free_energy}
\end{align*}

and also how heat capacity $C_V$ at constant volume relates to the partition function:

\begin{align*}
C_V &= \frac{1}{k_B T^2} (\langle E^2 \rangle - \langle E \rangle^2) \numberthis \label{eq:heat_capacity}
\end{align*}

We will also need the magnetic susceptibility, $\chi$, of the system:

\begin{align*}
\chi &= \frac{1}{k_B T} ( \langle \mathcal{M}^2 \rangle - \langle \mathcal{M} \rangle^2) \numberthis \label{eq:magnetic_susceptibility}
\end{align*}

In the following section we look at a sample system.


\subsubsection{Periodic \( 2 \times 2 \) square lattice with no external field}

We look at a system where we have particles in a 2x2 square lattice. A configuration of the system can be visualized as follows: \newline

\begin{center}
\begin{tabular}{cc}
$\bullet$ & $\bullet$ \\
$\bullet$ & $\bullet$
\end{tabular}  , \newline
\end{center}

where we change the dots out for arrows to mark the spin of the particles. A sample configuration of the system can be visualized as follows: \newline

\begin{center}
\begin{tabular}{cc}
$\uparrow$ & $\downarrow$ \\
$\downarrow$ & $\uparrow$
\end{tabular}  . \newline
\end{center}


We can denote the spins of the system as $s_{ij}$, where the index $i$ corresponds to the row the particle is in and $j$ to the column. With periodic boundary conditions, the total energy of the system is then given by:

\begin{align*}
E &= -J ( 2s_{11}s_{12} + 2s_{11}s_{21} + 2s_{22}s_{21} + 2s_{22}s_{12}) \numberthis \label{eq:2x2_sq_lattice_en}
\end{align*}

As there are two possible states that the particles can be in, and there are four particles total, there are $2^4 = 16$ microstates. We are interested in the possible energies of this system, and they are listed in table \ref{table:2x2_sq_lattice_en_and_mag} along with the total magnetization, where we have used \eqref{eq:2x2_sq_lattice_en} and \eqref{eq:ising_magnetization} to calculate these values. Note that the values in this table are sorted by macrostate, as some of the microstates are effectively equal because of the symmetries of the system. There are a total of 6 configurations with two spins pointing upwards, but these do not all have the same energy, which means that the macrostate of the system cannot be uniquely determined by the amount of spins that point up (or down for that matter). If the two spins pointing upwards are adjacent the total energy is 0, and there are four such configurations. If the two spins are not adjacent the total energy is $8J$, and there are two such states. As there are then two macrostates (different energies) corresponding to two spins pointing up, we need a second identifier. The degeneracy of these macrostates differ, and so by listing both the degeneracy and the amount of spins pointing up we can uniquely determine all the macrostates of the system. This was used in the table.

\begin{table}[H]
\centering
\caption{This table contains the energies and total magnetization for all possible macrostates of the Ising model with a 2x2 square lattice of particles. The uniqueness of the macro state is determined by the amount of spins that are up (first column) and the degeneracy of said state (second column). Both are needed, as in the case with two spins pointing up there are four configurations with the two upwards pointing spins being adjacent, and two with them not being adjacent.} \label{table:2x2_sq_lattice_en_and_mag}
\begin{tabular}{|c|c|c|c|}
\hline
Spins up & Degeneracy & Energy & Magnetization \\
\hline
4 & 1 & -8J & 4 \\
3 & 4 & 0 & 2 \\
2 & 4 & 0 & 0 \\
2 & 2 & 8J & 0 \\
1 & 4 & 0 & -2 \\
0 & 1 & -8J & -4 \\
\hline
\end{tabular}
\end{table}


We can calculate the partition function of the system using \eqref{eq:ising_partition_function}:

\begin{align*}
Z &= \sum\limits_i^M e^{-\beta E_i} \\
&= 2 \bigg(e^{8J\beta} + e^{-8J\beta} \bigg) + 12 \\
&= 4\cosh(8J\beta) + 12
\end{align*}

The probability distribution of the system is thus:

\begin{align*}
P_i(\beta) &= \frac{e^{-\beta E_i}}{4\cosh (8J\beta) + 12}
\end{align*}

The expectation value for the energy can be found as follows:

\begin{align*}
\langle E \rangle &= \sum\limits_i^M E_i P_i(\beta) \\
 &= -8J \frac{2e^{8J\beta}}{4\cosh (8J\beta) + 12} + 8J \frac{2e^{-8J}}{4\cosh (8J\beta) + 12} \\
 &= -8J\frac{2(e^{8J\beta} - e^{-8J\beta})}{4\cosh (8J\beta) + 12} \\
 &= -8J\frac{\sinh(8J\beta) }{\cosh (8J\beta) + 3} \numberthis \label{eq:2x2_energy}
\end{align*}

It is simple to see that $\langle M \rangle = 0$, so a more interesting measure would be the expectation value of the absolute of the magnetization. The expectation value for the absolute value of the magnetization can be found similarly:

\begin{align*}
\langle |\mathcal{M}| \rangle &= \sum\limits_i^M |\mathcal{M}_i| P_i(\beta) \\
 &= \frac{2e^{8J\beta} + 4}{\cosh (8 J \beta) + 3} \numberthis \label{eq:2x2_abs_magnetization}
\end{align*}

In order to find the heat capacity it is first necessary to find the expectation value of squared energy:

\begin{align*}
\langle E^2 \rangle &= \sum\limits_i^M E_i^2 P_i(\beta) \\
&= 64J^2 \frac{\cosh(8J\beta)}{\cosh(8J\beta) + 3}
\end{align*}

This gives us the heat capacity by equation \eqref{eq:heat_capacity}:

\begin{align*}
C_V &= \frac{1}{k_B T^2} ( \langle E^2 \rangle - \langle E \rangle^2) \\
&= \frac{64J^2}{k_B T^2} \frac{e^{-8J\beta}}{\cosh(8J\beta) + 3} \numberthis \label{eq:2x2_heat_capacity}
\end{align*}

Lastly we also want to find the magnetic susceptibility of the system. In order to do that we need to find:

\begin{align*}
\langle \mathcal{M}^2 \rangle &= \sum\limits_i \mathcal{M}_i^2 P_i(\beta) \\
&= 8 \frac{e^{8J\beta} + 1}{\cosh(8J\beta) + 3}
\end{align*}

This gives us the susceptibility by using equation \eqref{eq:magnetic_susceptibility}:

\begin{align*}
\chi &= \frac{1}{k_B T} ( \langle \mathcal{M}^2 \rangle - \langle \mathcal{M} \rangle^2 ) \\
&= \frac{8}{k_B T} \frac{e^{8J \beta}}{\cosh( 8J \beta) + 3} \numberthis \label{eq:2x2_susceptibility}
\end{align*}

All of these values we have calculated here can be compared with numerical results later on in order to evaluate the veracity of the numerical results.

The difference in energy between two states of a two dimensional lattice, can be calculated using \eqref{eq:ising_general_total_energy}. Assuming no outside magnetic field gives us the difference between states \(1\) and \(2\) as
\begin{align*}
	\Delta E = E_{2} - E_{1} = \left(-J \sum\limits_{<kl>}^N s_k^{2} s_l^{2}\right) -
	\left(-J \sum\limits_{<kl>}^N s_k^{1} s_l^{1}\right),
\end{align*}
where a superscript is used to denote which state the respective spins belong to.

Since we are only interested in the difference in energy from flipping a single spin in the lattice, we can assume that \(s_{k}^{1} = s_{k}^{2} = s_{k}\). Thus we can rewrite \(\Delta E\) as
\begin{align*}
 	\Delta E = J \sum_{<kl>}^{N} s_{k} \left(s_{l}^{1} - s_{l}^{2}\right).
\end{align*}

Furthermore, any spin can only be \(\pm 1\) and spin \(s_{l}\) is flipped such that \(s_{l}^{2} = -s_{l}^{1}\). From this it follows that \begin{align*}
	\left(s_{l}^{1} - s_{l}^{2}\right) = \left(s_{l}^{1} + s_{l}^{1}\right) = 2s_{l}^{1}.
\end{align*}

The difference in energy can then be written as

\begin{align*}
 	\Delta E = 2Js_{l}^{1} \sum_{<k>}^{N} s_{k}, \numberthis \label{eq:delta_e}
\end{align*}

where we only sum over \(k\), since only one spin \(s_{l}\) is flipped.
For a two dimensional lattice, each spin only has four neighbours, thus we can write out the sum for any randomly chosen \(s_{l}\) with neighbours \(s_{a}, s_{b}, s_{c}, s_{d}\) as

\begin{align*}
	\Delta E = 2Js_{l}^{1}\left(s_{a} + s_{b} + s_{c} + s_{d}\right) \\
	\Delta E = 2J(\pm 1)\left[(\pm 1) + (\pm 1) + (\pm 1) + (\pm 1)\right]
\end{align*}

From this we see that if all spins \(s_{k}\) have the same sign we get
\begin{align*}
	\Delta E = \pm 2J(\pm 4) = \pm 8J.
\end{align*}
If three of the spins \(s_{k}\) have the same sign we get
\begin{align*}
	\Delta E = \pm 2J(\pm 2) = \pm 4J.
\end{align*}
If two of the spins \(s_{k}\) have the same sign we get
\begin{align*}
	\Delta E = \pm 2J(0) = 0.
\end{align*}
Thus we only have five possible diffrences in energy between states when we only flip one spin.

The difference in magnetization can be simillarly found using \eqref{eq:ising_magnetization}, and subtracting \(M_{2} - M_{1}\)
\begin{align*}
	\mathcal{M}_2 - \mathcal{M}_{1} &= \sum\limits_j^N s_j^{2} - \sum\limits_j^N s_j^{1} =
	\sum\limits_{j}^{N}\left(s_{j}^{2} - s_{j}^{1}\right).
\end{align*}
Because we assume only spin \(s_{l}\) is flipped when changing states, the difference in magnetization becomes
\begin{align*}
 	\mathcal{M}_2 - \mathcal{M}_{1} = \left(s_{l}^{2} - s_{l}^{1}\right).
\end{align*} 
Using that \(s_{l}^{1} = - s_{l}^{2}\), we get
\begin{align*}
	\mathcal{M}_2 - \mathcal{M}_{1} = \left(s_{l}^{2} + s_{l}^{2}\right) = 2s_{l}^{2},
\end{align*}
which gives that
\begin{align*}
	\mathcal{M}_2 = \mathcal{M}_1 + 2s_{l}^{2} \numberthis \label{eq:delta_m}
\end{align*}


\subsubsection{Periodic square lattice with no external field in the thermodynamical limit} \label{sec:II:a:ii}

Solving the two-dimensional Ising model is a highly non-trivial task, and so we will only list the resulting expressions in the case where there is no external field. These expressions were found by Lars Onsager so we refer to his work \citep{L.Onsager1944} instead of performing the calculations ourselves. The partition function is given by the following:

\begin{align*}
Z_N &= [2\cosh(\beta J) e^I ]^N \, , \numberthis \label{eq:LO_partition_function}
\end{align*}

where: 

\begin{align*}
I &= \frac{1}{2\pi} \int_0^\pi d\phi \, \ln \bigg[\frac{1}{2}\bigg( 1 + ( 1 - \kappa^2 \sin^2 \phi)^{1/2} \bigg) \bigg] \, , 
\end{align*}

and:

\begin{align*}
\kappa &= \frac{2\sinh ( 2\beta J) }{\cosh^2(2\beta J)} \, ,
\end{align*}

and $N$ is the number of particles. This determines the mean energy:

\begin{align*}
\langle E \rangle &= -J \coth( 2\beta J) \bigg[ 1 + \frac{2}{\pi} (2\tanh^2 (2\beta J) - 1) K_1(\kappa) \bigg] \, , \numberthis \label{eq:LO_energy}
\end{align*}

where:

\begin{align*}
K_1 (\kappa) &= \int_0^{\pi/2} \frac{d\phi}{\sqrt{1 - \kappa^2 \sin^2 \phi}} \, .
\end{align*}

The heat capacity is derived from this quantity again, and is found to be:

\begin{align*}
C_V &= \frac{4k_B}{\pi} (\beta J \coth(2\beta J))^2 \bigg( K_1(\kappa) - K_2(\kappa) \\ 
& \quad - (1 - \tanh^2 (2\beta J) ) \bigg[ \frac{\pi}{2} + (2\tanh^2 (2\beta J) - 1)K_1(\kappa) \bigg] \bigg) \, , \numberthis \label{eq:LO_heat_capacity}
\end{align*}

where:

\begin{align*}
K_2(\kappa) &= \int_0^{\pi/2} d\phi \, \sqrt{1 - \kappa^2 \sin^2 \phi}
\end{align*}

The mean magnetization per spin is given by:

\begin{align*}
\langle \mathcal{M} / N \rangle &= \bigg[1 - \frac{(1- \tanh^2 (\beta J) )^4}{16\tanh^4 (\beta J)} \bigg]^{1/8} \, , \numberthis \label{eq:LO_magnetization_per_spin}
\end{align*}

when the temperature is lower than a specific critical value $T_C$. Above this value the mean magnetization per spin is zero.

These solutions have the interesting feature that they predict a phase transition at this critical temperature $T_C$. The heat capacity, magnetization and susceptibility all excibit a power law behaviour (\citep{Stanley1999}, \cite[chapter 2.1.2]{landau_binder_2014}) in the region where $T\to T_C$ and $T$ is close to $T_C$. Specifically the heat capacity behaves as:

\begin{align*}
C_V &\sim \bigg| T_C - T \bigg| ^{-\alpha} \, , \numberthis \label{eq:power_law_heat_capacity}
\end{align*} 

when $T\to T_C$ from below, and $T$ is close to $T_C$. It can be determined that the correct solution is $\alpha = 0$, meaning that this behaves according to the power law singularity in the thermodynamic limit. The magnetization per spin behaves as:

\begin{align*}
\langle \mathcal{M} / N \rangle &\sim (T_C - T)^{\beta} \, , \numberthis \label{eq:power_law_magnetization_per_spin}
\end{align*}

in the same region, and with $\beta \to 1/8$. Similarly the susceptibility can be shown to behave as:

\begin{align*}
\chi &\sim |T_C - T|^{-\gamma} \, , \numberthis \label{eq:power_law_susceptibility}
\end{align*}

in the same region, and with $\gamma \to 7/4$. The correlation function, $C(r)$, where the parameter $r$ defines the distance between spins, is a measure of how much the spins correlate with each other, and can be shown \citep{Stanley1999} to scale as:

\begin{align*}
C(r) &\sim  e^{-r/\xi} \, ,
\end{align*}

where $\xi$ is the correlation length. As we approach the critical temperature we expect this to decay slower, as a change in a spin should quickly correlate to a change in another spin elsewhere. As $r$ is only dependent on how far apart the spins are the only variable that can cause this to happen is the correlation length, meaning that it has to get large as we approach the critical temperature. Because of this it can be shown that close to the critical temperature the correlation length also exhibits power law behaviour  (this is shown in \citep{Stanley1999}, albeit by a different route than the one outlined here):

\begin{align*}
\xi (T) &\sim |T-T_C|^{-\nu} \, , \numberthis \label{eq:power_law_correlation_length_} 
\end{align*}

where $\nu$ is a constant. A phase transition such as the one that the Ising model predicts is characterized by a correlation length that spans the whole system, and thus this correlation length can also be connected to the size of the system if we are looking at a finite lattice:

\begin{align*}
\xi (T) &\propto L \, ,
\end{align*}

where $L$ is the amount of spins in one direction in this finite lattice. Connecting this with the previous result we can define a connection between the critical temperature in a finite lattice and the one in the thermodynamical limit:

\begin{align*}
T_C(L) - T_C(L= \infty) &\sim L^{-1/\nu} \, , \numberthis \label{eq:crit_temperature_finite_lattice} 
\end{align*}

where $\nu$ is the same constant as earlier. Combining this with the results for the heat capacity, the mean magnetization, and the magnetic susceptibility gives us the following relations in a finite lattice (\citep[p.78]{landau_binder_2014}):

\begin{align*}
C_V &\sim |T_C - T |^{-\alpha} \propto  L^{\alpha/\nu} \, , \numberthis \label{eq:finite_lattice_heat_capacity} \\
\langle \mathcal M \rangle &\sim (T-T_C)^\beta \propto L^{-\beta /\nu} \, , \numberthis \label{eq:finite_lattice_magnetization_per_spin} \\
\chi &\sim |T_C - T|^{-\gamma} \propto L^{\gamma/\nu} \, , \numberthis \label{eq:finite_lattice_susceptibility}
\end{align*}

where all variables are as defined earlier. We can use these relations to generate an estimate of $T_C(L)$, and if this is done for multiple $L$ we can estimate $T_C(L=\infty)$ by using equation \eqref{eq:crit_temperature_finite_lattice}. This allows us to calculate values for finite lattices numerically and relate them with results from the thermodynamic limit, which is essential as simulating an infinite is simply not possible. The analytical critical temperature with $\nu=1$ is (\citep{L.Onsager1944}):

\begin{align*}
\tilde{T}_C &= \frac{k_B T_C}{J} = \frac{2}{\ln(1 + \sqrt{2})} \approx 2.269 \, , \numberthis \label{eq:crit_temperature_analytical}
\end{align*}

where we have scaled the temperature $\tilde{T}_C$ using $k_B$ and $J$ so that it is a dimensionless quantity.


\subsection{The Ising model and the Metropolis algorithm} \label{sec:II:b}

One method of simulating a system in the Ising model is through a Markov Chain Monte Carlo setup, where we iterate and flip spins depending on the change in total energy that flipping said spins incur. The Metropolis algorithm \citep{Metropolis} is one such setup which we will explain in this section.

We propose a change in a set of spins will cause a change in the total configuration of the system. Changing the configuration of the system from one (denoted $a$) to another (denoted $b$) incurs a change in the total energy defined by $\Delta E = E_b - E_a$. If the change in energy is negative we instantly accept the new configuration, as the second law of thermodynamics states that the system should always move towards equillibrium (lowest energy) in a closed system. If however the change is positive (meaning that the system, we wish to let probabilities determine whether or not we accept the new configuration. We can calculate the probability of moving from configuration $a$ to $b$ by using the Boltzmann distribution \ref{eq:boltzmann_dist} ($P_i(T)$ where $i$ denotes the configuration of the system):

\begin{align*}
P_{a\to b}(T)  &= \frac{P_b (T) }{P_a (T)} = e^{-\beta \Delta E }\, . 
\end{align*}

In this case we assume that the change in energy is positive number, so this is always smaller than one as we wish for a probability. We then draw a random number $r$ between $0$ and $1$ using a random number generator, and compare this to the probability of the transition $P_{a\to b}(T)$. If $r\leq P_{a\to b} (T)$ we accept the new configuration, otherwise we keep the previous configuration. The process up to this point can be called a Monte Carlo cycle. The Monte Carlo cycle is performed a sufficient amount of times until the system has stabilized, which is characterized by the macroscopic variables of the system being stable.

As we can see we need to somehow calculate $P_{a\to b} (T)$ in every Monte Carlo cycle, and as this contains an exponential this is computationally expensive. We can, however, modify the Monte Carlo cycle in such a way that there only are set values of $\Delta E$, which means that we can precalculate $P_{a\to b} (T)$. Doing this will cause a massive speed up in the calculation. We have not yet outlined how we propose a new configuration, and this will also be covered by this same method.

If we flip only one spin at a time, there are only five possible changes in energy $\Delta E$. This can be shown by realizing that a spin only reacts with its closest neighbours. Only ten transitions are possible, but half of them are equal to the other half as they are equivalent in terms of change in energy. Ee only display those with a unique change in energy:

\begin{align*}
E = -4J \quad \begin{array}{ccc}
& \uparrow & \\
\uparrow & \uparrow & \uparrow \\
& \uparrow &
\end{array} \quad \to \quad E = 4J \quad \begin{array}{ccc}
& \uparrow & \\
\uparrow & \downarrow & \uparrow \\
& \uparrow &
\end{array} \, ,
\end{align*}

with $\Delta E = 8J$, 

\begin{align*}
E = -2J \quad \begin{array}{ccc}
& \uparrow & \\
\downarrow & \uparrow & \uparrow \\
& \uparrow &
\end{array} \quad \to \quad E = 2J \quad \begin{array}{ccc}
& \uparrow & \\
\downarrow & \downarrow & \uparrow \\
& \uparrow &
\end{array} \, ,
\end{align*}

with $\Delta E = 4J$,

\begin{align*}
E = 0 \quad \, \quad \begin{array}{ccc}
& \uparrow & \\
\downarrow & \uparrow & \uparrow \\
& \downarrow &
\end{array} \quad \to \quad E = 0 \,\,\,\, \quad \begin{array}{ccc}
& \uparrow & \\
\downarrow & \downarrow & \uparrow \\
& \downarrow &
\end{array} \, ,
\end{align*}

with $\Delta E = 0$, 

\begin{align*}
E = 2J \quad \begin{array}{ccc}
& \downarrow & \\
\downarrow & \uparrow & \uparrow \\
& \downarrow &
\end{array} \quad \to \quad E = -2J \quad \begin{array}{ccc}
& \downarrow & \\
\downarrow & \downarrow & \uparrow \\
& \downarrow &
\end{array} \, ,
\end{align*}

with $\Delta E = -4J$, and lastly:

\begin{align*}
E = 4J \quad \begin{array}{ccc}
& \downarrow & \\
\downarrow & \uparrow & \downarrow \\
& \downarrow &
\end{array} \quad \to \quad E = -4J \quad \begin{array}{ccc}
& \downarrow & \\
\downarrow & \downarrow & \downarrow \\
& \downarrow &
\end{array} \, ,
\end{align*}

with $\Delta E = -8J$. These are the only possible changes in energy. The other five possible transitions are the ones where every spin in the constellations above are flipped, but the change in energy is equivalent in these cases and so they are not specifically listed. With this in mind we can precalculate all the probabilities $P_{a \to b} (T)$ and use these. In the case where $\Delta E < 0$ we have that $P_{a\to b} (T) > 1$, which means that it seizes to be a proper probability, but if we draw a random number and compare it to this as outlined earlier, the test will always pass if $\Delta E < 0$, meaning that we do not have to handle this as a special case. In a two-dimensional system it is beneficiary to denote the spins as matrix elements $s_{ij}$ where $i$ is the lattice position of the spin in the $x$-direction and similarly for $j$ in the $y$-direction. With this notation we can establish an expression for the energy change from flipping spin $s_{ij}$:

\begin{align*}
\Delta E &= -2J s_{ij} \bigg( s_{i,j+1} + s_{i,j-1} + s_{i+1,j} + s_{i-1,j} \bigg) \numberthis \label{eq:singleflip_energy_change}
\end{align*}

A problem we encounter at this point is how we model the periodic boundary conditions. If we use a square lattice, we can use the amount of spins in a dimension $L$ and the modulo operator to define a function $f(i,k)$ that takes a valid index $i$ and what is to be added to it $k$, and returns the correct index on the other side of the lattice if it is out of bounds:

\begin{align*}
f(i) &= (i+k+L)\%L \, , 
\end{align*}

where \% is the modulo operator in this case. Note that we add $L$ to the total as well, as the function could return a negative value if $k$ was negative otherwise. The function might still return a negative value, but we have assumed here that $|k|$ is never larger than $L$. 
 
We have established that flipping one spin at a time is beneficial to the speed of the calculation. For larger systems this is a very small change, and so it is a good idea to flip more than one spin in each Monte Carlo cycle. Also if we want a more random process each cycle, we can also pick random spins that we attempt to flip, which we also choose to do later on. Exactly how all this is performed is outlined in section \hyperref[sec:III]{III}.

Every time a spin is flipped we need to update total energy and magnetization. The change in energy is $\Delta E$ and this can simply be added to the total. In the case of the magnetization, flipping a spin will change the total magnetization by two times the flipped value of the spin. Flipping a spin up means changing it's spin-value by $2$ (from $-1$ to $1$ gives $\Delta s = 1 - (-1) = 2$) and similarly flipping a spin down means changing it's spin-value by $-2$. 



\newpage

\section{Method} \label{sec:III}

In this report we present simulations of the Ising model, in two dimensions with no external field, using the Metropolis algorithm. We have chosen to implement our simulations in C++, with framework to operate said simulations in Python, with some post-processing of data also being performed in Python. In this section we will first discuss details on how particular calculations are performed, and then go into more detail on the layout of our implementation. 

\subsection{Details on calculations} \label{sec:III:a}

\subsubsection{Initialization of system} \label{sec:III:a:i}

Firstly, we will discuss how the system is initialized. We implemented two ways of initializing the spin matrix, total energy and magnetization. They are generated as follows:

\begin{cpp}
// Initiate spin matrix and magnetization
if (randspin) {
  srand(time(NULL));
  for (int x = 0; x < n_spins; x++) {
    for (int y = 0; y < n_spins; y++) {
      spin_matrix(x,y) = (rand() > RAND_MAX/2) ? -1 : 1;
      M += spin_matrix(x,y);
    }
  }
} else {
  spin_matrix.ones();
  M = n_spins2;
}


// Initial energy
for(int y =0; y < n_spins; y++) {
  for (int x= 0; x < n_spins; x++){
    E -=  double(spin_matrix(y,x)*
          (spin_matrix(periodic(y,n_spins,-1),x) +         
  	      spin_matrix(y,periodic(x,n_spins,-1))));
  }
}
\end{cpp}

Depending on the boolean variable \verb+randspin+ the spin-matrix is either generated as having all spins up (if \verb+randspin+ is false) or having a random spin configuration (if \verb+randspin+ is true). The spin matrix is an Armadillo \citep{Armadillo} integer matrix, which lets us use the \verb+ones()+ function to set all the spins to one. In this case the magnetization is simply the same as the total amount of spins which is stores in the variable \verb+n_spins2+. The random configuration is generated by using a double for loop, and random values drawn from the standard library function \verb+rand()+. In this case, every spin value is added to the variable that stores the magnetization (\verb+M+) in the same double for loop. The energy cannot be calculated at the same time as the spin matrix is being generated, as it needs all the neares neighbours to be generated already. Thus this calculation is separated into a secondary for loop, which is used in both cases. In general calculation of energy and magnetization is based on equations  \eqref{eq:ising_general_total_energy} and \eqref{eq:ising_magnetization} respectively.


\subsubsection{Performing a Monte Carlo sweep} \label{sec:III:a:ii}

The following code snippet performs one Monte Carlo sweep (flipping several spins, one at a time) on the system:

\begin{cpp}
std::random_device rd;
std::mt19937_64 gen(rd());
std::uniform_real_distribution<double> RNG(0.0, 1.0);
// Loop over all spins
for(int y =0; y < n_spins; y++) {
  for (int x= 0; x < n_spins; x++){

    // Get random indices
    int ix = int(RNG(gen)*(double)n_spins);
    int iy = int(RNG(gen)*(double)n_spins);

    // Calculate change in energy
    int deltaE =  2*spin_matrix(iy,ix)*
                  (spin_matrix(iy,periodic(ix,n_spins,-1))+
                  spin_matrix(periodic(iy,n_spins,-1),ix) +
                  spin_matrix(iy,periodic(ix,n_spins,1)) +
                  spin_matrix(periodic(iy,n_spins,1),ix));


    // Flip spin if new config is accepted
    if ( RNG(gen) <= w(deltaE+8) ) {
      spin_matrix(iy,ix) *= -1;


      // Update energy and magnetization if spin is flipped
      M += double(2*spin_matrix(iy,ix));
      E += double(deltaE);

      // Count accepted configs
      accepted_configs++;
    }
  }
}
\end{cpp}


\subsection{Implementation} \label{sec:III:b}


\newpage

\section{Results} \label{sec:IV}


\newpage

\section{Discussion} \label{sec:V}


\newpage

\section{Conclusion} \label{sec:VI}


\onecolumngrid
\bibliography{kilder.bib}{}
\newpage
\twocolumngrid

\appendix
\section{Source code} \label{A}
All code for this report was written in C++ and Python 3.8, and the complete set of files can be found at
\url{https://github.com/eivinsto/FYS3150_Project_X.git}.

\newpage
\section{Selected results} \label{B}
Here is a folder of selected results from running our code.

\url{https://github.com/eivinsto/FYS3150_Project_X/tree/master/data}

~
\newpage
\section{System specifications} \label{C}
All results included in this report were achieved by running the implementation on the following system:

\begin{itemize}
	\item CPU: AMD Ryzen \(9\) \(3900\)X
	\item RAM: \(2\times\SI{8}{\giga\byte}\) Corsair Vengeance LPX DDR\(4\) \(\SI{3200}{\mega\hertz}\)
\end{itemize}

\end{document}
